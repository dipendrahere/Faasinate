
\documentclass{article}
\usepackage[hmargin=1in,vmargin=1in]{geometry}
\usepackage{listings}
\usepackage{color}
\IfFileExists{lmodern.sty}{\usepackage{lmodern}}{}
\usepackage[T1]{fontenc}
\usepackage[utf8]{inputenc}
\lstset{
    numbers=left,               
    numberstyle=\small \ttfamily \color[rgb]{0.4,0.4,0.4},
    showspaces=false,               
    showstringspaces=false,         
    showtabs=false,                 
    frame=lines,                    
    tabsize=4,                      
    breaklines=true,                
    breakatwhitespace=false,        
    basicstyle=\ttfamily,   
    keywordstyle=\color[rgb]{0.133,0.133,0.6},
    commentstyle=\color[rgb]{0.133,0.545,0.133},
    stringstyle=\color[rgb]{0.627,0.126,0.941},
}
\begin{document}


\begin{lstlisting}[language=Java]
package com.example.dipendra.paygetping;

import android.app.ProgressDialog;
import android.content.Intent;
import android.content.SharedPreferences;
import android.os.Bundle;
import android.preference.PreferenceManager;
import android.support.annotation.NonNull;
import android.support.v7.app.AppCompatActivity;
import android.util.Log;
import android.view.Menu;
import android.view.MenuItem;
import android.widget.Toast;

import com.example.dipendra.paygetping.login.LoginActivity;
import com.example.dipendra.paygetping.login.RegisterActivity;
import com.example.dipendra.paygetping.models.User;
import com.example.dipendra.paygetping.models.WalletListWrapper;
import com.example.dipendra.paygetping.utils.Constants;
import com.example.dipendra.paygetping.wallet.AddUserToList;
import com.example.dipendra.paygetping.wallet.WalletsActivity;
import com.google.android.gms.tasks.OnCompleteListener;
import com.google.android.gms.tasks.Task;
import com.google.firebase.auth.AuthResult;
import com.google.firebase.auth.FirebaseAuth;
import com.google.firebase.auth.FirebaseUser;
import com.google.firebase.database.DataSnapshot;
import com.google.firebase.database.DatabaseError;
import com.google.firebase.database.DatabaseReference;
import com.google.firebase.database.ValueEventListener;

/**
 * Created by dipendra on 06/04/17.
 */

public class BaseActivity extends AppCompatActivity {
    protected FirebaseAuth auth;
    protected ProgressDialog progressDialog;
    protected SharedPreferences sp;
    protected FirebaseAuth.AuthStateListener authStateListener;
    protected User user;

    public User getUser() {
        return user;
    }

    public WalletListWrapper getCurrentList() {
        return currentList;
    }

    protected WalletListWrapper currentList;
    protected DatabaseReference reference;
    @Override
    protected void onCreate(Bundle savedInstanceState) {
        super.onCreate(savedInstanceState);
        auth = FirebaseAuth.getInstance();
        authStateListener = new FirebaseAuth.AuthStateListener() {
            @Override
            public void onAuthStateChanged(@NonNull FirebaseAuth firebaseAuth) {
                FirebaseUser firebaseUser = auth.getCurrentUser();
                if(firebaseUser == null){
                    getToLoginScreen();
                }
                else{
                    readSharedPreferences();
                    try {
                        if (currentList.getOwnerID() != null && currentList.getOwnerID().equals(user.getEncodedEmail()) && BaseActivity.this instanceof MainActivity) {
                            Constants.menu = R.menu.menu_main;
                        }
                        else  {
                            Constants.menu = R.menu.menu_main_1;
                        }
                    }catch (Exception e){
                        Log.e("TAG:: MAIN ACTIVITY", "Menu not yet created");
                    }
                    invalidateOptionsMenu();
                }
            }
        };
        initialize();
    }
    private void initialize(){
        sp = PreferenceManager.getDefaultSharedPreferences(this);
        user = new User();
        currentList = new WalletListWrapper();
        progressDialog = new ProgressDialog(this);
        progressDialog.setIndeterminate(true);
        progressDialog.setCancelable(false);
        if(Constants.isOnline(this) == false){
            Toast.makeText(this, "Working Off-the line is not good! Please be online!", Toast.LENGTH_SHORT).show();
        }
    }
    private void getToLoginScreen(){
        Intent i = new Intent(BaseActivity.this, LoginActivity.class);
        i.setFlags(Intent.FLAG_ACTIVITY_NEW_TASK | Intent.FLAG_ACTIVITY_CLEAR_TASK);
        startActivity(i);
    }
    protected void readSharedPreferences(){
        sp = PreferenceManager.getDefaultSharedPreferences(this);
        user.setEncodedEmail(sp.getString("encodedEmail", null));
        user.setName(sp.getString("username", null));
        String nameOfTheCurrentList = sp.getString("currentWalletName", null);
        if(nameOfTheCurrentList == null && !(this instanceof WalletsActivity)){
            Intent i = new Intent(this, WalletsActivity.class);
            i.setFlags(Intent.FLAG_ACTIVITY_NEW_TASK | Intent.FLAG_ACTIVITY_CLEAR_TASK);
            startActivity(i);
        }
        else{
            currentList.setName(nameOfTheCurrentList);
        }
        currentList.setOwner(sp.getString("currentWalletOwner", null));
        currentList.setPushId(sp.getString("currentWallet", null));
        currentList.setOwnerID(sp.getString("currentWalletOwnerID", null));
        try {
            if (this instanceof MainActivity) {
                getSupportActionBar().setTitle("Wallet : "+ currentList.getName());
            }
        }
        catch (Exception e){
            e.printStackTrace();
        }
       // Toast.makeText(this, "You are:"+ user.getName()+user.getEncodedEmail(), Toast.LENGTH_LONG).show();
    }
    @Override
    protected void onStart() {
        super.onStart();
        if(!(this instanceof LoginActivity || this instanceof RegisterActivity)){
            auth.addAuthStateListener(authStateListener);
        }
    }

    @Override
    protected void onResume() {
        super.onResume();
    }



    @Override
    protected void onStop() {
        super.onStop();
        if(authStateListener != null){
            if(!(this instanceof LoginActivity || this instanceof RegisterActivity)) {
                auth.removeAuthStateListener(authStateListener);
            }
        }
    }
    protected void signin(String usermail, String passwordString){
        progressDialog = new ProgressDialog(this);
        progressDialog.setMessage("Signing in.. Please Wait..");
        progressDialog.setCancelable(false);
        progressDialog.setIndeterminate(true);
        progressDialog.show();
        final String mail = usermail.replace(".",",");
        auth.signInWithEmailAndPassword(usermail, passwordString).addOnCompleteListener(this, new OnCompleteListener<AuthResult>() {
            @Override
            public void onComplete(@NonNull Task<AuthResult> task) {
                if(!task.isSuccessful()){
                    progressDialog.hide();
                    try{
                        throw task.getException();
                    } catch (Exception e) {
                        Toast.makeText(BaseActivity.this, "Either Username or password is wrong",Toast.LENGTH_LONG).show();
                    }
                }
                else{
                    final SharedPreferences.Editor ed = sp.edit();
                    reference = Constants.getDatabase().getReference().child("users").child(mail);
                    reference.addListenerForSingleValueEvent(new ValueEventListener() {
                        @Override
                        public void onDataChange(DataSnapshot dataSnapshot) {
                            user = dataSnapshot.getValue(User.class);
                            ed.putString("username", user.getName());
                            ed.putString("encodedEmail", user.getEncodedEmail());
                            ed.commit();
                            Intent i = new Intent(BaseActivity.this, MainActivity.class);
                            i.setFlags(Intent.FLAG_ACTIVITY_NEW_TASK | Intent.FLAG_ACTIVITY_CLEAR_TASK);
                            progressDialog.hide();
                            startActivity(i);
                        }

                        @Override
                        public void onCancelled(DatabaseError databaseError) {
                            progressDialog.hide();
                            Toast.makeText(BaseActivity.this, "Some Error Occured",Toast.LENGTH_SHORT).show();
                        }
                    });
                }
            }
        });
    }

    @Override
    public boolean onCreateOptionsMenu(Menu menu) {
        getMenuInflater().inflate(Constants.menu, menu);
        return true;
    }
    @Override
    public boolean onOptionsItemSelected(MenuItem item) {
        int id = item.getItemId();
        if(id == R.id.logout){
            signOutUser();
        }
        else if(id == R.id.add_user_to_list){
            Intent i = new Intent(BaseActivity.this, AddUserToList.class);
            startActivity(i);
        }
        return super.onOptionsItemSelected(item);
    }
    protected void signOutUser(){
        auth.signOut();
        SharedPreferences.Editor editor = sp.edit();
        editor.putString("encodedEmail", null);
        editor.putString("username", null);
        editor.putString("currentWallet", null);
        editor.putString("currentWalletName", null);
        editor.putString("currentWalletOwner", null);
        editor.commit();
        user = null;
        getToLoginScreen();
    }

}

\end{lstlisting}

\end{document}
                     mark2[(lc)+1] = 1;
      //                  lazy2[node].p2 = lazy2[node].p5 = 0;
		}
		lazy2[node].p2 = lazy2[node].p5 = 0;
		tree[node].r = 0;
                return;
        }
        int m = (s+e)/2;
        update2((lc),s,m,l,r,p2,p5);
        update2(rc,m+1,e,l,r,p2,p5);
        tree[node] = merge(tree[(lc)],tree[rc],tree[node].r);
}
struct node query(int node,int s,int e,int l,int r)
{
	int lc = (node<<1),rc = lc+1;
/*	if(tree[node].r == 1){
		if(s != e){
			lazy[lc].p2 = lazy[lc].p5 = lazy[rc].p2 = lazy[rc].p5 = 0;
            tree[lc].r = 1;
            tree[lc].p2 = tree[lc].p5 = tree[(lc)+1].p2 = tree[(lc)+1].p5 = 0;
            mark1[lc] = mark1[rc] = 0;
            tree[rc].r = 1;
		}
		tree[node].r = 0;
//		cout << "q : "<< s << " " << e << " " << pp2 << " " << pp5 << " " << endl;
	}*/
	if(s > e || s > r || e < l){
                struct node tm;
                tm.p2 = tm.p5 = tm.r = 0;
                return tm;
        }
	if(mark2[node] != 0){
  //              mark1[node] = 0;
                tree[node].p2 = (lazy2[node].p2*(e-s+1));
                tree[node].p5 = (lazy2[node].p5*(e-s+1));
 /*               n = (e-lazy2[node].l + 1);
                t2 = 2;
                t5 = 5;
                pp2 = pp5 = 0;
                while(n/t2){
                        pp2 += (n/t2);
                        t2 *= 2;
                }
                while(n/t5){
                        pp5 += (n/t5);
                        t5 *= 5;
                }
                n = (s-lazy2[node].l);
                t2 = 2;
                t5 = 5;
               // pp2 = pp5 = 0;
                while(n/t2){
                        pp2 -= (n/t2);
                        t2 *= 2;
                }
                while(n/t5){
                        pp5 -= (n/t5);
                        t5 *= 5;
                }*/
		t2 = (e-lazy2[node].l + 1);
		t5 = (s- lazy2[node].l);
		pp2 = an[t2].p2 - an[t5].p2;
		pp5 = an[t2].p5 - an[t5].p5;
                tree[node].p2 += pp2;
                tree[node].p5 += pp5;
//                cout << "mark q2 :" << s << " " << e << " " << tree[node].p2 << " " << tree[node].p5 << endl;
                if(s != e){
                		lazy[lc].p2 = lazy[lc].p5 = lazy[rc].p2 = lazy[rc].p5 = 0;
                       	mark1[lc] = mark1[rc] = 0;
                        lazy2[(lc)].p2 = lazy2[node].p2;
                        lazy2[rc].p5 = lazy2[node].p5;
                        lazy2[(lc)].p5 = lazy2[node].p5;
                        lazy2[rc].p2 = lazy2[node].p2;
                        lazy2[(lc)].l = lazy2[node].l;
                        lazy2[rc].l = lazy2[node].l;
                        mark2[lc] = 1;
                        mark2[(lc)+1] = 1;
                }
                lazy2[node].p2 = lazy2[node].p5 = 0;
                mark2[node] = 0;
	}
	if(mark1[node] != 0){
//				cout << "q : "<< s << " " << e << " " << pp2 << " " << pp5 << " " << endl;
                tree[node].p2 += (lazy[node].p2*(e-s+1));
                tree[node].p5 += (lazy[node].p5*(e-s+1));
                if(s != e){
                        lazy[(lc)].p2 += lazy[node].p2;
                        lazy[rc].p5 += lazy[node].p5;
                        lazy[(lc)].p5 += lazy[node].p5;
                        lazy[rc].p2 += lazy[node].p2;
                        mark1[lc] = 1;
                        mark1[(lc)+1] = 1;
                }
                lazy[node].p2 = lazy[node].p5 = 0;
                mark1[node] = 0;
        }
	if(l <= s && e <= r){
//		cout << "qqmark :" << s << " " << e << " " << tree[node].p2 << " " << tree[node].p5 << endl;
		return tree[node];
	}
	/*if(s > e || s > r || e < l){
                struct node tm; 
                tm.p2 = tm.p5 = tm.r = 0;
                return tm;
        }*/
	int m = (s+e)>>1;
	if(r <= m)
		return query((lc),s,m,l,r);
	else if(l > m)
		return query(rc,m+1,e,l,r);
	return merge(query((lc),s,m,l,r) , query(rc,m+1,e,l,r),0);
}
int main()
{	int n2,c,t,i,j,v,op,x,y;
	long long sum = 0;
	struct node ans;
	scanint(t);
	for(i = 2;i <= 100001;i++){
		n = i;
		t2 = 2;
                t5 = 5;
                pp2 = pp5 = 0;
                while(n/t2){
                        pp2 += (n/t2);
                        t2 *= 2;
                }
                while(n/t5){
                        pp5 += (n/t5);
                        t5 *= 5;
                }
/*                n = (s-lazy2[node].l);
                t2 = 2;
                t5 = 5;
               // pp2 = pp5 = 0;
                while(n/t2){
                        pp2 -= (n/t2);
                        t2 *= 2;
                }
                while(n/t5){
                        pp5 -= (n/t5);
                        t5 *= 5;
                }*/
		an[i].p2 = pp2;
		an[i].p5 = pp5;
	}
	while(t--){
	//	cout << "t : " << t << endl;
		sum = 0;
		scanint(n2);
		scanint(c);
		for(i = 0;i < 400005 ;i++){
			tree[i].p2 = tree[i].p5 = tree[i].r = 0;
			mark2[i] = mark1[i] = 0;
			lazy[i].p2 = lazy[i].p5 = lazy2[i].p2 = lazy2[i].p5 = lazy2[i].l = 0;
		}
		for(i = 1;i <= n2;i++){
			scanint(a[i]);
		}
		build(1,1,n2);	
		while(c--){
	//		cout <<"c : "  << c << " " << endl;
			scanint(op);
			if(op == 1){
				scanint(i);
				scanint(j);
				scanint(x);
			//	cout << c << endl;
				n = x;
                		pp2 = pp5 = 0;
                		while(n%2 == 0){
                        		pp2++;
                        		n = n>>1;
                		}
                		while(n%5 == 0){
                        		pp5++;
                        		n /= 5;
                		}
//				cout << pp2 << pp5 << endl;
				update1(1,1,n2,i,j,pp2,pp5);
			}
			else if(op == 2){
				scanint(i);
				scanint(j);
				scanint(y);
			//	cout << c << endl;
				n = y;
                                pp2 = pp5 = 0;
                                while(n%2 == 0){
                                        pp2++;
                                        n = n/2;
                                }
                                while(n%5 == 0){
                                        pp5++;
                                        n /= 5;
                                }
                                update2(1,1,n2,i,j,pp2,pp5);
			}
			else{
			//	cout << c << endl;
				scanint(i);
				scanint(j);
				ans = query(1,1,n2,i,j);
	//			cout << ans.p2 << " " <<ans.p5 << endl;
				sum += min(ans.p2,ans.p5);
			}
		}
		printf("%lld\n",sum);
	}
}

\end{lstlisting}

\end{document}
